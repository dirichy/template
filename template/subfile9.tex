%!TEX TS-program = xelatex
%!language = zh
\documentclass[main]{subfiles}
%这是一个子文件,单独编译时会自动导入main文件的导言区
%这里可以放自定义命令,不会和别人的冲突请放心
%但是不能放newtheorem等高级命令,需要请在群里说
\newcommand{\gal}{\mathrm{Gal}}
\newcommand{\id}{\mathop{\mathrm{id}}}
\renewcommand{\char}{\mathop{\mathrm{char}}}
\begin{document}
\renewcommand{\filename}{No.9Theorem}%在这里填你的文件名,避免\label冲突
\section{高次方程一般没有根式解}
\subsection{背景介绍}
解代数方程一直是数学中的重要问题.对于二次方程 \(ax^2+bx+c=0,a \neq 0\)而言,我们很容易能解出其两个根为 \(x= \frac{-b \pm \sqrt{b^2-4ac}}{2a}\).
事实上,古巴比伦留下的陶片显示,在大约公元前2000年(2000 BC)古巴比伦的数学家就能解一元二次方程了.
对于三次方程 \(ax^3+bx^2+cx+d=0,a \neq 0\)而言,由于涉及到复数开根,所以其一般解被发现的较晚.
一般而言我们认为是尼科洛·塔尔塔利亚最早在1553年发现了三次方程的一般解.
三次方程的根式解复杂程度比二次方程提高了很多,其中包括了开三次根、复数等运算.
四次方程的根式解在三次方程根式解出现不久后就被人们发现,但是其解的复杂程度非常巨大,已经几乎失去了实用价值,只有理论研究的作用.

当二次、三次、四次方程的根式解被得到后,数学家当然不会满足,他们开始向五次方程挑战.
人们相信五次方程根式解的出现只是时间问题,不管它有多复杂,总会被人们发现.
然而无论数学家如何改进解方程的方法,都无法在与五次方程斗争的路上前进哪怕一小步.
终于,在1824年,阿贝尔证明了一般的五次方程没有根式解.
之后,伽罗瓦创造性地引入了群这一数学概念,对方程是否存在根式解给出了一个具体的刻画.
至此,解代数方程的这一挑战才画上(不那么完美的)句号.
\subsection{定理叙述}
\begin{theorem}\label{the:1}
  五次及以上的代数方程一般没有根式解.
\end{theorem}
所谓一般没有根式解,是指在绝大多数情况下都没有根式解,或者说没有统一的根式解的形式,而不是说所有的高次方程都一定没有根式解.
比如\(x^5=1\) 显然有根式解\(x=\sqrt[5]{1}\).
对于绝大多数高次方程,如\(x^5-x-1=0\) 而言,都是没有根式解的.
\section{证明概述}
伽罗瓦创造性地引入了群的概念来证明高次

\end{document}
