%!TEX TS-program = xelatex
%!language = zh
\documentclass[main]{subfiles}
%这是一个子文件,单独编译时会自动导入main文件的导言区
%这里可以放自定义命令,不会和别人的冲突请放心
%但是不能放newtheorem等高级命令,需要请在群里说
\newcommand{\gal}{\mathrm{Gal}}
\newcommand{\id}{\mathop{\mathrm{id}}}
\renewcommand{\char}{\mathop{\mathrm{char}}}
\begin{document}
\renewcommand{\filename}{No.9Theorem}%在这里填你的文件名,避免\label冲突
\section{高次方程一般没有根式解}
\subsection{背景介绍}
解代数方程一直是数学中的重要问题.对于二次方程 \(ax^2+bx+c=0,a \neq 0\)而言,我们很容易能解出其两个根为 \(x= \frac{-b \pm \sqrt{b^2-4ac}}{2a}\).
事实上,古巴比伦留下的陶片显示,在大约公元前2000年(2000 BC)古巴比伦的数学家就能解一元二次方程了.
对于三次方程 \(ax^3+bx^2+cx+d=0,a \neq 0\)而言,由于涉及到复数开根,所以其一般解被发现的较晚.
一般而言我们认为是尼科洛·塔尔塔利亚最早在1553年发现了三次方程的一般解.
三次方程的根式解复杂程度比二次方程提高了很多,其中包括了开三次根、复数等运算.
四次方程的根式解在三次方程根式解出现不久后就被人们发现,但是其解的复杂程度非常巨大,已经几乎失去了实用价值,只有理论研究的作用.

当二次、三次、四次方程的根式解被得到后,数学家当然不会满足,他们开始向五次方程挑战.
人们相信五次方程根式解的出现只是时间问题,不管它有多复杂,总会被人们发现.
然而无论数学家如何改进解方程的方法,都无法在与五次方程斗争的路上前进哪怕一小步.
终于,在1824年,阿贝尔证明了一般的五次方程没有根式解.
之后,伽罗瓦创造性地引入了群这一数学概念,对方程是否存在根式解给出了一个具体的刻画.
至此,解代数方程的这一挑战才画上(不那么完美的)句号.
\subsection{定理叙述}
\begin{theorem}\label{the:1}
  五次及以上的代数方程一般没有根式解.
\end{theorem}
所谓一般没有根式解,是指在绝大多数情况下都没有根式解,或者说没有统一的根式解的形式,而不是说所有的高次方程都一定没有根式解.
比如\(x^5=1\) 显然有根式解\(x=\sqrt[5]{1}\).
对于绝大多数高次方程,如\(x^5-x-1=0\) 而言,都是没有根式解的.
\section{证明概述}
伽罗瓦创造性地引入了群(Group)的概念来证明高次方程没有根式解。所谓的群就是一个用来描述对称性的代数结构。
比如全体整数就是一个群,可以用来描述整数轴在平移下的对称性,因为保持整数轴不变的平移方法刚好是平移\(n\)格,其中\(n\)是整数。

一般而言,一个体系对称性越好,其对称群元素就越多。
比如正方形在旋转下的对称群有四个元素(旋转\(\frac{k \pi}{4} \mathrm{rad},k=0,1,2,3\));
而旋转对称性更好的圆的对称群则有无数个元素。

一个体系的对称性越好,我们就越难从其中确定一个元素。
在正方形中,我们要找一个特定的点,如顶点,只需在四个顶点中挑出这个点即可。
由于其对称群是有限的,我们总是可以把要找的点限定在一个有限集合中。
但是在圆中,想确定一个点却很困难。一个圆上的所有点都是没有区别的,我们很难找到一个特定的点。

伽罗瓦发现,一个代数方程的根的复杂程度也可以用群来描述。具体而言,一个代数方程的根的复杂程度取决于交换某些根的位置能否被代数系统``发现''。
如\(x^2-2=0\)的两个根\(\pm \sqrt{2}\)就无法由四则运算和有理数区分。如果\((a+b \sqrt{2})(c+d \sqrt{2})=e+f \sqrt{2}\),那么同样有\((a-b \sqrt{2})(c-d \sqrt{2})=(e-f \sqrt{2})\)。
而\(x^4-1=0\)的四个根\(\pm 1,\pm \mathrm{i}\)就可以被区分,我们知道\(1*2=2\),但\(\mathrm{i}*2 \neq 2\)。

我们知道,对称性越强,就越难具体表达。根式解作为一种表达方式,其能处理的对称性也是有限的。
具体而言,伽罗瓦定义了一个多项式的所有根的置换中无法被四则运算和有理数发现的部分为这个多项式的伽罗瓦群,
然后发现多项式是否可根式解只和这个群有关。
当这个群比较简单,如\(x^2-2=0\)的伽罗瓦群中只有两个元素,这个多项式就可以根式解,对应的群也被称为可解群。
当这个群很复杂时,这个多项式就不可解,对应的群就叫不可解群。
如\(x^5-x-1=0\),它的五个根任意交换位置都不能被初等代数系统发现,因此其伽罗瓦群很复杂,有\(5!=120\)个元素,从而不能根式解。

\end{document}
