%!TEX TS-program = xelatex
%!language = zh
\documentclass[main]{subfiles}
%这是一个子文件,单独编译时会自动导入main文件的导言区
%这里可以放自定义命令,不会和别人的冲突请放心
%但是不能放newtheorem等高级命令,需要请在群里说
\newcommand{\gal}{\mathrm{Gal}}
\newcommand{\id}{\mathop{\mathrm{id}}}
\renewcommand{\char}{\mathop{\mathrm{char}}}
\begin{document}
\renewcommand{\filename}{No.9Theorem}%在这里填你的文件名,避免\label冲突
\section{高次方程一般没有根式解}
\subsection{背景介绍}
解代数方程一直是数学中的重要问题.对于二次方程\(ax^2+bx+c=0,a \neq 0\)而言,我们很容易能解出其两个根为\(x= \frac{-b \pm \sqrt{b^2-4ac}}{2a}\).
事实上,古巴比伦留下的陶片显示,在大约公元前2000年(2000 BC)古巴比伦的数学家就能解一元二次方程了.
对于三次方程\(ax^3+bx^2+cx+d=0,a \neq 0\)而言,由于涉及到复数开根,所以其一般解被发现的较晚.
一般而言我们认为是尼科洛·塔尔塔利亚最早在1553年发现了三次方程的一般解.
三次方程的根式解复杂程度比二次方程提高了很多,其中包括了开三次根、复数等运算.
四次方程的根式解在三次方程根式解出现不久后就被人们发现,但是其解的复杂程度非常巨大,已经几乎失去了实用价值,只有理论研究的作用.

当二次、三次、四次方程的根式解被得到后,数学家当然不会满足,他们开始向五次方程挑战.
人们相信五次方程根式解的出现只是时间问题,不管它有多复杂,总会被人们发现.
然而无论数学家如何改进解方程的方法,都无法在与五次方程斗争的路上前进哪怕一小步.
终于,在1824年,阿贝尔证明了一般的五次方程没有根式解.
之后,伽罗瓦创造性地引入了群这一数学概念,对方程是否存在根式解给出了一个具体的刻画.
至此,解代数方程的这一挑战才画上(不那么完美的)句号.
\subsection{定理叙述}
要想证明高次方程无根式解,我们首先需要定义什么叫根式解.所谓根式解就是用加、减、乘、除、开\(n\)次根号进行有限次迭代的解.
具体而言,我们有
\begin{defination}\label{def:1}
	设\(f( x) \in K[ x]\)是一个多项式,其中\(K\)是一个域.若存在域扩张链
	\[
		K=K_0 \subset K_1 \subset K_2 \subset \cdots \subset K_m
	\]
	满足\(K_m\)包含\(f\)的所有根,且对于\(t=0,\cdots,m-1\),有\(K_{t+1}=K_t[ a]\),其中\(a\)满足\(\exists n \in \mathbb{N},a^n \in K_t\).
	则称\(f\)有根式解,或者\(f\)根式可解.
\end{defination}
阿贝尔发现五次方程一般没有根式解,即
\begin{theorem}\label{the:1}
	存在一个五次多项式\(f\)使得\(f\)根式不可解.
\end{theorem}
伽罗瓦在描述可解性的时候引入了伽罗瓦群的概念.
\begin{defination}\label{def:2}
	设\(f( x) \in K[ x]\)为一个多项式,设\(F\)为\(f\)的分裂域.
	定义\(f\)的伽罗瓦群为\(\gal_K( f)=\{ \sigma :\sigma \text{是}F \text{的自同构,且}\sigma|_K=\id\}\).
\end{defination}
同时与可解多项式对应地定义了可解群.
\begin{defination}\label{def:3}
	若一个有限群\(G\)满足存在正规子群链\(G=G_m \unrhd G_{m-1} \unrhd \cdots \unrhd G_0\),且\(G_{t+1} / G_t\)都是循环群,则称\(G\)是可解群.
\end{defination}
伽罗瓦神奇地洞察了可解多项式与可解群的关系,即
\begin{theorem}\label{the:2}
	如果\(f\)可根式解,那么\(\gal_K( f)\)是可解群;
	如果\(\gal_K( f)\)是可解群,且域\(K\)的特征满足\(\char K \nmid \deg f\), 则\(f\)可根式解.
\end{theorem}
在大部分情况下五次以上的多项式的伽罗瓦群并不可解,因此不能根式解.
% \subsection{证明概述}
% 方便起见我们只考虑特征为\(0\)的域\(\mathbb{Q}\),其他的域证明方法类似.首先我们需要证明如下引理
% \begin{lemma}\label{lem:1}
% 	设\(E\)为\(f( x) \in \mathbb{Q}[ x]\)的分裂域,则域扩张\(E / \mathbb{Q}\)的中间域与\(\gal( f)\)的子群有如下一一对应关系:
% 	\[
% 		L \leftrightarrow H,L=\{ a \in E:\forall \sigma \in H,\sigma( x)=x\},H=\{ \sigma \in \gal( f):\forall a \in L,\sigma( a)=a\}
% 	\]
% 	记作\(L'=H,H'=L\).
% \end{lemma}
% \begin{proof}
% 	首先我们用归纳法可以证明若\(L \subset M\)是两个中间域,则有\([ L':M']\leq [ M:L]\).
% 	由Artin引理可得对于两个伽罗瓦群的子群\(J \leq H\), 有\([ H':J']\leq[ J:H]\).
% 	由定义再得到\(H \leq H''\)以及\(L \subset L''\),结合\(K'' =K\),我们就能得到这个一一对应关系.
% \end{proof}
% 由这个引理结合一些域扩张的技巧我们最终就能证明一个多项式可根式解等价与其伽罗瓦群可解.
% 下面我们来证明五次多项式一般不可根式解.
% \begin{corollary}
% 	\(x^5-x-1=0\)不可根式解.
% \end{corollary}
% \begin{proof}
% 	设其五个根为\(x_1,x_2,x_3,x_4,x_5\).考虑\(S_5\)到\(\gal( x^5-x-1)\)的同态:
% 	\[
% 		\theta( \sigma)( f( x_1,x_2,x_3,x_4,x_5)):=f( x_{\sigma( 1)},x_{\sigma( x)},x_{\sigma( 3)},x_{\sigma( 4)},x_{\sigma( 5)}).
% 	\]
% 	不难验证\(\theta\)是一个同构,于是我们得到\(\gal( x^5-x-1) \cong S_5\).
% 	由于\(S_5\)的非平凡正规子群只有\(A_5\),而\(A_5\)的换位子群为\(A_5\),且\(A_5\)不是交换群.故\(A_5\)不可解,从而\(S_5\)也不可解.
% \end{proof}

\end{document}
