%!TEX TS-program = xelatex
\documentclass[main]{subfiles}
%这是一个子文件,单独编译时会自动导入main文件的导言区
%这里可以放自定义命令,不会和别人的冲突请放心
%但是不能放newtheorem等高级命令,需要请在群里说
%下面是一些数学命令的简化,可以保留,可以删去,也可以按你的习惯修改
\newcommand{\mx}{\mathrm{d}x}
\newcommand{\my}{\mathrm{d}y}
\newcommand{\mz}{\mathrm{d}z}
\newcommand{\mr}{\mathrm{d}r}
\newcommand{\mmu}{\mathrm{d}u}
\newcommand{\mv}{\mathrm{d}v}
\newcommand{\mte}{\mathrm{d}\theta}
\newcommand{\mfai}{\mathrm{d}\varphi}
\newcommand{\df}{\dfrac}
\newcommand{\tf}{\tfrac}
\newcommand{\pa}{\partial}
\newcommand{\di}{\displaystyle}
\newcommand{\q}{$}
\newcommand{\bo}{\boldsymbol}
\newcommand{\te}{\theta}
\newcommand{\fai}{\varphi}
\newcommand{\cd}{\cdot}
\newcommand{\sq}{\sqrt}
\newcommand{\uns}{\underset}
\newcommand{\Om}{\Omega}
\newcommand{\ti}{\times}
\newcommand{\la}{\lambda}
\newcommand{\ora}{\overrightarrow}
\newcommand{\al}{\alpha}
\newcommand{\be}{\beta}
\newcommand{\ga}{\gamma}
\newcommand{\gt}{\geqslant}
\newcommand{\lt}{\leqslant}
\newcommand{\bs}{\boldsymbol}
\newcommand{\va}{\varepsilion}
\newcommand{\bb}{\mathbb}

\begin{document}
\renewcommand{\filename}{subfile15}%在这里填你的文件名,避免\label冲突
%这里开始写你的代码
%\title{No. 15 二项式定理}\renewcommand\maketitlehookc{\vspace{-18ex}}\date{}

%\maketitle
\section{二项式定理}
\begin{theorem}
    设$n$是正整数,$x$和 $y$是实数,则可以将 $x+y$ 的$n$次幂展开成和的形式
    $$
    (x+y)^n=\left(\begin{array}{l}n \\
    0
    \end{array}\right) x^n y^0+\left(\begin{array}{c}
    n \\
    1
    \end{array}\right) x^{n-1} y^1+\left(\begin{array}{c}
    n \\
    2
    \end{array}\right) x^{n-2} y^2+\cdots+\left(\begin{array}{c}
    n \\
    n-1
    \end{array}\right) x^1 y^{n-1}+\left(\begin{array}{l}
    n \\
    n
    \end{array}\right) x^0 y^n
    $$
    
    其中每个 $\left(\begin{array}{c}n \\k \end{array}\right)$
    为一个称作二项式系数的特定正整数,其等于 $\frac{n !}{k !(n-k) !}$.
\end{theorem}
二项式定理 %(英语: Binomial theorem) 
描述了二项式的幂的代数展开.根据该定理,可以将两个数之和的整数次幂诸如 $(x+y)^n$ 
展开为类似 $a x^b y^c$ 项之和的恒等式,
其中 $b$、$c$ 均为非负整数且 $b+c=n$.系数 $a$ 是依赖于 $n$ 和 $b$ 的正整数.
当某项的指数为 0 时,通常略去不写.例如: 
$$
(x+y)^4=x^4+4 x^3 y+6 x^2 y^2+4 x y^3+y^4 \text {. }
$$
$a x^b y^c$ 中的系数 $a$ 被称为二项式系数,记作 
$\left(\begin{array}{l}n \\ b\end{array}\right)$ 
或 $\left(\begin{array}{l}n \\ c\end{array}\right)$ 
(二者值相等).二项式定理可以推广到任意实数次幂,即广义二项式定理.

\iffalse 
\begin{proof}
        用数学归纳法证明. 当 $n=1$ 时,
       $$
       (a+b)^1=\sum_{k=0}^1\left(\begin{array}{l}
       1 \\
       k
       \end{array}\right) a^{1-k} b^k=\left(\begin{array}{l}
       1 \\
       0
       \end{array}\right) a^1 b^0+\left(\begin{array}{l}
       1 \\
       1
       \end{array}\right) a^0 b^1=a+b.
       $$
       
       假设二项展开式在 $n=m$ 时成立.当 $n=m+1$时,
       $$
       \begin{aligned}
       (a+b)^{m+1} & =a(a+b)^m+b(a+b)^m \\
       & =a \sum_{k=0}^m\left(\begin{array}{c}
       m \\
       k
       \end{array}\right) a^{m-k} b^k+b \sum_{j=0}^m\left(\begin{array}{c}
       m \\
       j
       \end{array}\right) a^{m-j} b^j \\
       & =\sum_{k=0}^m\left(\begin{array}{c}
       m \\
       k
       \end{array}\right) a^{m-k+1} b^k+\sum_{j=0}^m\left(\begin{array}{c}
       m \\
       j
       \end{array}\right) a^{m-j} b^{j+1} \\
       & =a^{m+1}+\sum_{k=1}^m\left(\begin{array}{c}
       m \\
       k
       \end{array}\right) a^{m-k+1} b^k+\sum_{j=0}^m\left(\begin{array}{c}
       m \\
       j
       \end{array}\right) a^{m-j} b^{j+1}  \\
       & =a^{m+1}+\sum_{k=1}^m\left(\begin{array}{c}
       m \\
       k
       \end{array}\right) a^{m-k+1} b^k+\sum_{k=1}^{m+1}\left(\begin{array}{c}
       m \\
       k-1
       \end{array}\right) a^{m-k+1} b^k  \\
       & =a^{m+1}+\sum_{k=1}^m\left(\begin{array}{c}
       m \\
       k
       \end{array}\right) a^{m-k+1} b^k+\sum_{k=1}^m\left(\begin{array}{c}
       m \\
       k-1
       \end{array}\right) a^{m+1-k} b^k+b^{m+1}  \\
       & =a^{m+1}+b^{m+1}+\sum_{k=1}^m\left[\left(\begin{array}{c}
       m \\
       k
       \end{array}\right)+\left(\begin{array}{c}
       m \\
       k-1
       \end{array}\right)\right] a^{m+1-k} b^k  \\
       & =a^{m+1}+b^{m+1}+\sum_{k=1}^m\left(\begin{array}{c}
       m+1 \\
       k
       \end{array}\right) a^{m+1-k} b^k  \\
       & =\sum_{k=0}^{m+1}\left(\begin{array}{c}
       m+1 \\
       k
       \end{array}\right) a^{m+1-k} b^k
       \end{aligned}
       $$
       由归纳法可知,定理成立.
\end{proof}
\fi

二项式系数的三角形排列通常被认为是法国数学家布莱兹·帕斯卡的贡献,
他在17世纪描述了这一现象.但早在他之前,就曾有数学家进行类似的研究.
例如,古希腊数学家欧几里得于公元前4世纪提到了指数为2的情况.
公元前三世纪,印度数学家青目探讨了更高阶的情况.
帕斯卡三角形的雏形于10世纪由印度数学家大力罗摩发现.
在同一时期,波斯数学家卡拉吉和数学家兼诗人欧玛尔·海亚姆得到了
更为普遍的二项式定理的形式.13世纪,中国数学家杨辉也得到了类似的结果.
卡拉吉用数学归纳法的原始形式给出了二项式定理和帕斯卡三角形(巴斯卡三角形)
的有关证明.艾萨克·牛顿勋爵将二项式定理的系数推广到有理数.
  

\end{document}
