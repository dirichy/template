%!TEX TS-program = xelatex
\documentclass[main]{subfiles}
%这是一个子文件,单独编译时会自动导入main文件的导言区
%这里可以放自定义命令,不会和别人的冲突请放心
%但是不能放newtheorem等高级命令,需要请在群里说
%下面是一些数学命令的简化,可以保留,可以删去,也可以按你的习惯修改
\newcommand{\mx}{\mathrm{d}x}
\newcommand{\my}{\mathrm{d}y}
\newcommand{\mz}{\mathrm{d}z}
\newcommand{\mr}{\mathrm{d}r}
\newcommand{\mmu}{\mathrm{d}u}
\newcommand{\mv}{\mathrm{d}v}
\newcommand{\mte}{\mathrm{d}\theta}
\newcommand{\mfai}{\mathrm{d}\varphi}
\newcommand{\df}{\dfrac}
\newcommand{\tf}{\tfrac}
\newcommand{\pa}{\partial}
\newcommand{\di}{\displaystyle}
\newcommand{\q}{$}
\newcommand{\bo}{\boldsymbol}
\newcommand{\te}{\theta}
\newcommand{\fai}{\varphi}
\newcommand{\cd}{\cdot}
\newcommand{\sq}{\sqrt}
\newcommand{\uns}{\underset}
\newcommand{\Om}{\Omega}
\newcommand{\ti}{\times}
\newcommand{\la}{\lambda}
\newcommand{\ora}{\overrightarrow}
\newcommand{\al}{\alpha}
\newcommand{\be}{\beta}
\newcommand{\ga}{\gamma}
\newcommand{\gt}{\geqslant}
\newcommand{\lt}{\leqslant}
\newcommand{\bs}{\boldsymbol}
\newcommand{\va}{\varepsilion}
\newcommand{\bb}{\mathbb}

\begin{document}
\renewcommand{\filename}{subfile20}%在这里填你的文件名,避免\label冲突
%这里开始写你的代码
%\title{No. 20 拉格朗日定理}\renewcommand\maketitlehookc{\vspace{-18ex}}\date{}

%\maketitle
\section{拉格朗日定理}
在群论中,拉格朗日定理表明了对任何有限群$G$,每个$G$的子群的阶(元素个数)整除$G$的阶,
该定理以约瑟夫-路易-拉格朗日命名.
%In the mathematical field of group theory, Lagrange's theorem is a theorem that states that for any finite group $G$, 
%the order (number of elements) of every subgroup of $G$ divides the order of $G$. The theorem is named after Joseph-Louis Lagrange. 
定理进一步表明,对于有限群 $G$ 的子群 $H$而言,$|G| /|H|$ 不仅是个整数,而且它的值等于指标 $[G: H]$,其中 $[G: H]$ 是 $H$ 在 $G$ 中左陪集的个数.
%The following variant states that for a subgroup $H$ of a finite group $G$, not only is $|G| /|H|$ an integer,
%but its value is the index $[G: H]$, defined as the number of left cosets of $H$ in $G$.
\begin{theorem}[拉格朗日定理]
    设$G$是有限群,若 $H$ 是 $G$ 的子群,那么$|G|=[G: H] \cdot|H|$.
\end{theorem}

如果将 $|G|,|H|$, 和 $[G: H]$ 看作基数,那么这个定理对 $G$ 是无限阶群的情况也是成立的.
%This variant holds even if $G$ is infinite, 
%provided that $|G|,|H|$, and $[G: H]$ are interpreted as cardinal numbers.
\begin{proof}
    规定 $G$ 中的元素 $x$ 与 $y$ 等价如果存在 $h\in H$,使得 $x=yh$,这是一个等价关系.
    于是 $H$ 在 $G$ 中的左陪集便是在此等价关系中的等价类.因此,左陪集构成了 $G$ 的一个分划.每个左陪集 $aH$ 有与
     $H$ 相同的基数,因为 $x \mapsto a x$ 定义了从 $H \rightarrow a H$ 的一个双射.而左陪集的数量是指标 $[G: H]$,
     综上所述,
     $$
     |G|=[G: H] \cdot|H| .
     $$
\end{proof}
%Proof 
%The left cosets of $H$ in $G$ are the equivalence classes of a certain equivalence relation on $G$ :
 %specifically, 
%call $x$ and $y$ in $G$ equivalent if there exists $h$ in $H$ such that $x=y h$. 
%Therefore, the left cosets form a partition of $G$. Each left coset $a H$ has the same 
%cardinality as $H$ because $x \mapsto a x$ defines a bijection $H \rightarrow a H$ (the inverse is $y \mapsto a^{-1} y$ ). 
%The number of left cosets is the index $[G: H]$. By the previous three sentences,
%$$
%|G|=[G: H] \cdot|H| .
%$$

\subsection{应用}
这个定理的一个推论是指群中元素的阶整除群的阶:若群 $G$ 有 $n$ 个元素,$a\in G$,那么 $a^n=e$.

这个定理可以用来证明费马小定理以及它的一般化:欧拉定理.

这个定理也表明任何素数阶群 $G$ 是循环群,也是单群,
因为由任何非单位元生成的子群必须是群 $G$ 本身.
%This can be used to prove Fermat's little theorem and its generalization, 
%Euler's theorem. 
%These special cases were known long before the general theorem was proved.

%The theorem also shows that any group of prime order is cyclic and simple, 
%since the subgroup generated by any non-identity element must be the 
%whole group itself.

拉格朗日定理还可以用来证明有无限多个素数:
假设存在一个最大的素数 $p$,那么梅森素数 $2^p-1$ 的任何素因子 $q$ 满足:
$2^p \equiv 1 \quad(\bmod q)$ ,意味着乘法群 
$(\mathbb{Z} / q \mathbb{Z})^*$ 中 $2$ 的阶是 $p$. 
由拉格朗日定理,$2$ 的阶必须整除 $(\mathbb{Z} / q \mathbb{Z})^*$ 的阶 $q-1$,
于是 $p$ 整除 $q-1$,所以 $p<q$,这与 $p$ 是最大的素数矛盾!
%Lagrange's theorem can also be used to show that there are infinitely
 %many primes: 
%suppose there were a largest prime $p$. Any prime divisor $q$ of the 
%
%$2^p-1$ satisfies $2^p \equiv 1 \quad(\bmod q)$ (see modular arithmetic),
  %meaning that the order of 2 in the multiplicative group 
 % $(\mathbb{Z} / q \mathbb{Z})^*$ is $p$. 
  %By Lagrange's theorem, the order of 2 must divide 
  %the order of $(\mathbb{Z} / q \mathbb{Z})^*$,
  % which is $q-1$. So $p$ divides $q-1$, giving $p<q$, 
 %  contradicting the assumption that $p$ is the 
  % largest prime. ${ }^{[2]}$

\end{document}
