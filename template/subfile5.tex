%!TEX TS-program = xelatex
\documentclass[main]{subfiles}
%这是一个子文件,单独编译时会自动导入main文件的导言区
%这里可以放自定义命令,不会和别人的冲突请放心
%但是不能放newtheorem等高级命令,需要请在群里说
%下面是一些数学命令的简化,可以保留,可以删去,也可以按你的习惯修改
\def\e{\textup{e}}
\def\i{\textup{i}}
\def\T{\textup{T}}
\def\diag{\textup{diag}}
\def\id{\textup{id}}
\newcommand{\toi}[1]{{#1}\to\infty}
\newcommand{\dis}{\displaystyle}
\newcommand{\bv}{\mathrm{BV}}
\newcommand{\ac}{\mathrm{AC}}
\newcommand{\mr}{\mathbb{R}}
\newcommand{\mn}{\mathbb{N}}
\newcommand{\mq}{\mathbb{Q}}
\newcommand{\mz}{\mathbb{Z}}
\newcommand{\rel}{\text{ rel }}
\newcommand{\sgn}{\operatorname{sign}}
\newcommand{\ve}{\varepsilon}
\newcommand{\bs}{\backslash}
\renewcommand{\ll}{\lim\limits}
\renewcommand{\span}{\operatorname{span}}
\renewcommand{\ker}{\operatorname{Ker}}
\renewcommand{\hom}{\operatorname{Hom}}
\renewcommand{\leq}{\leqslant}
\renewcommand{\geq}{\geqslant}

\begin{document}
\renewcommand{\filename}{二次互反律}%在这里填你的文件名,避免\label冲突
%这里开始写你的代码
\section{二次互反律 \\ Law of quadratic reciprocity}

\subsection{背景与定理内容}

二次方程$x^2-7=0$没有整数解和有理数解, 但有实数解$\pm\sqrt{7}$. 这启示我们, 如果将解的条件放宽 (这里是允许解存在于一个更大的环或域, 如整环$\mz[\sqrt{7}]$或实数域$\mr$), 方程的可解性可能会发生变化. 换个角度, 我们在模$p$ ($p$为素数)的意义下考虑上述方程整数解的存在性, 即允许系数与解变动$p$的若干整数倍. 例如, 取$p=3$, 有$4^2 = 16 \equiv 7 \pmod 3$, 从而在模$3$的意义下$x=4$是上述方程的一个解. 现在考虑一般的情形:

\begin{definition}[二次剩余]
	设 $p$ 为素数, $n$ 为与 $p$ 互素的整数.
	称 $n$ 为模 $p$ 的\textbf{二次剩余},
	如果存在整数 $m$ 使得
	 \(
	m^2 \equiv n \pmod p
	\)
	成立, 否则称 $n$ 为模 $p$ 的\textbf{二次非剩余}.
\end{definition}

为表征二次剩余与否, 我们定义 \textbf{Legendre 符号}
\[
	\left( \frac{n}{p} \right) := \left\{\begin{array}{ll}
		1,  & \mbox{若 $n$ 是模 $p$ 的二次剩余},  \\
		-1, & \mbox{若 $n$ 是模 $p$ 的二次非剩余}, \\
		0,  & \mbox{若 $n=0$}.
	\end{array}\right.
\]

两个(奇)素数之间的二次剩余满足如下关系, 称为\textbf{二次互反律}.

\begin{theorem}[二次互反律]
	对于奇素数$p \neq q$, 有
	\[\left( \frac{p}{q} \right) \left( \frac{q}{p} \right) = (-1)^{\frac{p-1}{2}\frac{q-1}{2}}.\]
\end{theorem}

\begin{theorem}[补充定理]
	对于奇素数$p$, 有 \(\left( \frac{-1}{p} \right) = (-1)^{\frac{p-1}{2}}, \left( \frac{2}{p} \right) = (-1)^{\frac{p^2-1}{8}}.\)
\end{theorem}

\subsection{应用举例}
二次互反律可以简化 Legendre 符号的计算, 尤其对于较大的素数. 在此之前, 我们需要列举 Legendre 符号的基本性质:

\begin{proposition}
	\begin{itemize}
		\item 设 $p$ 为素数, $a$ 为与 $p$ 互素的整数, $k$ 为任意整数, 有 \(\left( \frac{a}{p} \right) = \left( \frac{a + kp}{p} \right)\).
		\item \textup{乘性:} 设$p$为素数, $m, n$ 为与 $p$ 互素的整数, 有 \(\left( \frac{m}{p} \right) \left( \frac{n}{p} \right) = \left( \frac{mn}{p} \right)\).
		\item \textup{Euler 准则:} 设 $p$ 为奇素数, $a$ 为与 $p$ 互素的整数, 有 \(\left(\frac{a}{p}\right) \equiv a^{\frac{p-1}{2}} \pmod p\).
	\end{itemize}
\end{proposition}

第一条性质从 Legendre 符号的定义即可看出; 第二条性质说明为判断二次方程$x^2 \equiv n \pmod p$的可解性, 只需分别考虑 $n$ 的素因子; 第三条性质为 Legendre 符号的计算提供了一般方法.

\begin{example}[开始的例子]
	由\textup{Euler 准则}, \(\left( \frac{7}{3} \right) \equiv 7^{\frac{3-1}{2}} \equiv 1 \pmod 3\), 从而 $ 7 $ 是模 $ 3 $ 的二次剩余.
\end{example}

\begin{example}[大素数]
	$31$是模$103$的二次非剩余, 因为 \(\left(\frac{31}{103}\right)
	\stackrel{*}{=} -\left(\frac{103}{31}\right)
	= -\left(\frac{-21}{31}\right)
	= -\left(\frac{-1}{31}\right) \left(\frac{21}{31}\right)
	= \left(\frac{21}{31}\right)
	= \left(\frac{3}{31}\right) \left(\frac{7}{31}\right)
	\stackrel{*}{=} -\left(\frac{31}{3}\right) \cdot \left[- \left(\frac{31}{7}\right)\right]
	= \left(\frac{1}{3}\right) \left(\frac{3}{7}\right)
	\stackrel{*}{=} \left(\frac{1}{3}\right) \cdot \left[- \left(\frac{7}{3}\right)\right]
	= -1.\) 其中带$ * $的等号用到了二次互反律.
\end{example}

\end{document}
