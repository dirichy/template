%!TEX TS-program = xelatex
\documentclass[main]{subfiles}
%这是一个子文件,单独编译时会自动导入main文件的导言区
%这里可以放自定义命令,不会和别人的冲突请放心
%但是不能放newtheorem等高级命令,需要请在群里说
%下面是一些数学命令的简化,可以保留,可以删去,也可以按你的习惯修改
\def\e{\textup{e}}
\def\i{\textup{i}}
\def\T{\textup{T}}
\def\diag{\textup{diag}}
\def\id{\textup{id}}
\newcommand{\toi}[1]{{#1}\to\infty}
\newcommand{\dis}{\displaystyle}
\newcommand{\bv}{\mathrm{BV}}
\newcommand{\ac}{\mathrm{AC}}
\newcommand{\mr}{\mathbb{R}}
\newcommand{\mn}{\mathbb{N}}
\newcommand{\mq}{\mathbb{Q}}
\newcommand{\mz}{\mathbb{Z}}
\newcommand{\rel}{\text{ rel }}
\newcommand{\sgn}{\operatorname{sign}}
\newcommand{\ve}{\varepsilon}
\newcommand{\bs}{\backslash}
\newcommand{\Span}{\operatorname{span}}
\renewcommand{\ll}{\lim\limits}
\renewcommand{\ker}{\operatorname{Ker}}
\renewcommand{\hom}{\operatorname{Hom}}
\renewcommand{\leq}{\leqslant}
\renewcommand{\geq}{\geqslant}
\begin{document}
	\renewcommand{\filename}{26. 多边形外角和}%在这里填你的文件名,避免\label冲突
	%这里开始写你的代码
\section{26. 多边形外角和}
\subsection{定理叙述}
	在平面中, 任意凸多边形的外角和都是$360^{\circ}$.
	
	尽管这一结论看起来过于显然, 从多边形的任一顶点出发, 沿着多边形的边走一圈回到原点, 转过的角之和一定是$360^{\circ}$, 但这一结论并不是平面几何的公理, 仍然需要证明.
\subsection{定理证明}
	\begin{theorem}\label{key}
		在平面中, 若两直线平行, 则内错角相等.
	\end{theorem}
	\begin{proof}
		该定理就是欧几里得的《几何原本》\cite{Euclid}中的命题1.29, 其中关键之处在于用到了{\kaishu 第五条公理(平行公设)}: 若两条直线都与第三条直线相交, 并且在同一边的内角之和小于两个直角, 则这两条直线在这一边必定相交. 不过近现代的数学家认为欧几里得的公理体系并不是有严格的数理逻辑支撑的公理体系, 尤其是第五条平行公设导致了很多问题. 后来希尔伯特建立了平面几何新的公理体系\cite{Hilbert}, 在这一新体系下证明平行直线内错角相等主要是用到了其{\kaishu 合同(Congruence)公理的第4条}, 大致意思是给定一条射线和张角的大小后可唯一确定另一条射线, 具体证明可参考\cite{Hartshorne}的第113页.
	\end{proof}
	
	\begin{theorem}
		在平面中, 任意三角形的内角和都是$180^{\circ}$.
	\end{theorem}
	\begin{proof}
		欧几里得《几何原本》\cite{Euclid}的命题1.31以及希尔伯特公理体系\cite{Hilbert}的平行公理都认定过直线外一点可以作该直线的平行线.
	\end{proof}
	
	\begin{corollary}\label{sum of interior angles}
		在平面中, 任意凸$n$边形的内角和都是$(n-2)\cdot 180^{\circ}$.
	\end{corollary}
	
	\begin{corollary}\label{sum of exterior angles}
		在平面中, 任意凸多边形的外角和都是$360^{\circ}$.
	\end{corollary}
	\begin{proof}
		对于凸多边形而言, 每个顶点处内角、外角之和都是$180^{\circ}$, 那么$n$个顶点处的内外角之和就是$n\cdot 180^{\circ}$; 而推论\ref{sum of interior angles}指出内角和是$(n-2)\cdot 180^{\circ}$, 因此外角和就是$n\cdot 180^{\circ}-(n-2)\cdot 180^{\circ}=360^{\circ}$.
	\end{proof}
	
	\begin{remark}
		\textup{推论\ref{sum of interior angles}、\ref{sum of exterior angles}对于一般的多边形(即, 可能有大于$180^{\circ}$的内角)是否成立呢? 笔者暂未想出平面几何知识下严谨的证明, 读者可以尝试证明一下:)}
	\end{remark}
	
	{\small
		\begin{thebibliography}{9}
			\bibitem{Euclid} Fitzpatrick, R. (2008). Euclid's elements of geometry.
			\bibitem{Hilbert} Hilbert, D. (1902). The foundations of geometry. Open court publishing Company.
			\bibitem{Hartshorne} Hartshorne, R. (2013). Geometry: Euclid and beyond. Springer Science \& Business Media.
		\end{thebibliography}
	}
\end{document}