%!TEX TS-program = xelatex
%!language = zh
\documentclass[main]{subfiles}
%这是一个子文件,单独编译时会自动导入main文件的导言区
%这里可以放自定义命令,不会和别人的冲突请放心
%但是不能放newtheorem等高级命令,需要请在群里说
\begin{document}
\renewcommand{\filename}{No.10Theorem}%在这里填你的文件名,避免\label冲突
\section{四平方和定理}
\subsection{背景介绍}
四平方和定理说明每个正整数均可表示为4个整数的平方和.
它是费马多边形数定理和华林问题的特例.
1743年,瑞士数学家欧拉发现了一个著名的恒等式:
\((a^2+b^2+c^2+d^2)(x^2+y^2+z^2+w^2)=(ax+by+cz+dw)^2+(ay-bx+cw-dz)^2+(az-bw-cx+dy)^2+(aw+bz-cy-dx)^2\)
根据上述欧拉恒等式可知如果正整数\(m\)和\(n\)能表示为4个整数的平方和,
则其乘积\(mn\)也能表示为4个整数的平方和.
1751年,欧拉又得到了另一个一般的结果.
即对任意奇素数 \(p\),同余方程
\(x^2+y^2+1 \equiv 0\pmod p\)
必有一组整数解\(x,y\)满足\(0 \le x<\frac{p}{2}\),\(0 \le y<\frac{p}{2}\)

至此,证明四平方和定理所需的全部引理已经全部证明完毕.
此后,拉格朗日和欧拉分别在1770年和1773年作出最后的证明.

\subsection{定理叙述}
\begin{theorem}\label{the:1}
	设\(n \in \mathbb{N}^+\),则存在\(a,b,c,d \in \mathbb{N}\)使得\(n=a^2 +b^2 + c^2 + d^2\).
\end{theorem}
\subsection{证明概述}
% \begin{lemma}\label{lem:1}
% 	若所有素数都能写成四个整数的平方和,那么所有正整数都可以写成四个整数的平方和.
% \end{lemma}
% \begin{proof}
% 	由每个正整数都可以写为若干个素数的积,结合四平方和恒等式易知.
% \end{proof}
只需证明所有素数可以写为四平方和.
\(2=1^2 + 1^2 + 0^2 + 0^2\),因此只需证明奇质数可以表示成四个整数的平方和.
\begin{proof}
	根据欧拉1973年的结果,存在\(0<x,y<\frac{p}{2} \)使得\(x^2 + y^2 + 1^2 + 0^2=kp\).
	令\(m_0:=\min\{k \in \mathbb{N}^+:\exists x_i,i=1,2,3,4,kp=\sum_{i=1}^{4}x_i^2\}\).
	从\(0<x,y<\frac{p}{2}\)可知\(m_0 < p\).

	若\(m_0\)是偶数,且\(m_0 p = x_1^2 + x_2^2 + x_3^2 + x_4^2\).
	不失一般性设\(x_1,x_2\)的奇偶性相同,奇偶分析知\(x_3,x_4\)的奇偶性也相同,
	则\(x_1+x_2,x_1-x_2,x_3+x_4,x_3-x_4\)均为偶数.
	从而\(\frac{m_0}{2}p = \left(\frac{x_1+x_2}{2}\right)^2 + \left(\frac{x_1-x_2}{2}\right)^2 + \left(\frac{x_3+x_4}{2}\right)^2 + \left(\frac{x_3-x_4}{2}\right)^2\)
	但\(\frac{m_0}{2} < m_0\),与\(m_0\)的定义矛盾.

	现在用反证法证明\(m_0 = 1\).设\(m_0 > 1\).
	易知\(m_0\)不可整除\(x_1,x_2,x_3,x_4\)的最大公因数,否则\(m_0^2\)可整除\(m_0 p\),则得\(m_0\)是\(p\)的因数,但\(1 < m_0 < p\)且p为质数,矛盾.
	故存在不全为零、绝对值小于\(\frac{1}{2} m_0\)(注意\(m_0\)是奇数在此的重要性)的整数\(y_1,y_2,y_3,y_4\)使得
	\[
		y_i \equiv x_i \pmod{m_0},0 < \sum_{i=1}^{4} y_i^2 < 4 (\frac{1}{2} m_0 )^2 = m_0^2 ,\sum_{i=1}^{4} y_i^2 \equiv \sum_{i=1}^{4} x_i^2 \equiv 0 \pmod{m_0}
	\]
	从而 \(\sum_{i=1}^{4} y_i^2  = m_0 m_1\),其中\(m_1 \in \mathbb{N}^+,m_1 < m_0\).
	下证\(m_1 p\)可以表示成四平方和.
	令:
	\[
		\begin{cases}
			z_1=x_1 y_1 + x_2 y_2 + x_3 y_3 + x_4 y_4,z_2 = x_1 y_2 - x_2 y_1 + x_3 y_4 - x_4 y_3 \\
			z_3 = x_1 y_3 - x_2 y_4 -x_3 y_1 + x_4 y_2,z_4 = x_1 y_4 + x_2 y_3 -x_3 y_2 -x_4 y_1
		\end{cases}
	\]
	则有\(\sum_{i=1}^{4} z_i^2 = \sum_{i=1}^{4} y_i^2 \times \sum_{i=1}^{4} x_i^2\).
	且易知\(z_1 \equiv \sum_{i=1}^{4}x_i^2 \equiv 0 \pmod{m_0}\).
	又有\(z_2 \equiv x_1 x_2 - x_2 x_1 +x_3 x_4 - x_4 x_3 \equiv 0 \pmod{m_0}\),同理\(z_3,z_4 \equiv 0 \pmod{m_0}\).
	故\(z_1,z_2,z_3,z_4\)是\(m_0\)的倍数,令\(z_i = m_0 t_i,i=1,2,3,4\),则有
	\(m_0^2 \sum_{i=1}^{4} t_i^2 = m_0 m_1 m_0 p\),从而
	\(\sum_{i=1}^{4} t_i^2 = m_1 p < m_0 p\),与\(m_0\)的定义
	矛盾.
\end{proof}

\end{document}
