%!TEX TS-program = xelatex
%!language = zh
\documentclass[main]{subfiles}
%这是一个子文件,单独编译时会自动导入main文件的导言区
%这里可以放自定义命令,不会和别人的冲突请放心
%但是不能放newtheorem等高级命令,需要请在群里说
\begin{document}
\renewcommand{\filename}{No.17Theorem}%在这里填你的文件名,避免\label冲突
\section{ \(\pi\)是超越数}
\subsection{背景介绍}
超越一词最早在莱布尼兹1962年的一篇论文中出现,用来描述函数.
他证明了 \(\sin x\)是 \(x\)的超越函数.
现代超越数的概念最早由欧拉在18世纪提出.
1844年,刘维尔证明了超越数存在,并在1851年给出了第一个例子:
\(L_b:=\sum_{n=1}^{\infty} 10^{-n!}\).
1873年,埃尔米特证明了自然对数的底 \(\mathrm{e}\)是超越数.
1874年,康托证明了实数几乎是超越数,并给出了一个构造超越数的系统方法.
1882年,林德曼证明了 \(\pi\)是超越数.
后来,又有许多数,如 \(\mathrm{e}^\pi,2^{\sqrt{2}},\sin 1,\ln a,e^b\)等,其中 \(a\)是不为 \(1\)的正有理数, \(b\)是不为 \(0\)的代数数.
\subsection{定理叙述}
\begin{definition}\label{def:1}
	若一个复数 \(x \in \mathbb{C}\)不是任何一个非零有理系数多项式的根,则称其为超越数.
	否则称其为代数数.
\end{definition}
\begin{theorem}\label{the:1}
	\(\pi\)是超越数,即对于任何非零有理系数多项式 \(f(x)\),都有 \(f(\pi)\neq 0\).
\end{theorem}
\subsection{证明概述}
证明\(\pi\)是无理数很简单,但是证明其是超越数却很难.
我们将使用林德曼-魏尔斯特拉斯定理证明\(\pi\)是超越数,在此之前我们需要先做一些基础的准备:
\begin{definition}\label{def:2}
	设 \(\alpha_1,\cdots,\alpha_n \in \mathbb{C}\).若对于不全为 \(0\)的 \(b_1,\cdots,b_n \in \mathbb{Q}\),都有 \(\sum_{i=1}^{n} b_i \alpha_i \neq 0\),则称它们在 \(\mathbb{Q}\)上线性无关.
	从定义可以得到,一个不为 \(0\)的复数自己本身一定是线性无关的.

	设 \(y_1,\cdots,y_n \in \mathbb{C}\).若对于非 \(0\)的有理系数 \(n\)元多项式 \(f(x_1,\cdots,x_n)\),均有 \(f(y_1,\cdots,y_n) \neq 0\),则称它们在 \(\mathbb{Q}\)上代数无关.
	从定义可以得到,一个复数自己是代数无关的当且仅当这个复数是超越数.
\end{definition}
\begin{definition}\label{def:3}
	对于复数\(z \in \mathbb{C}\),令\(\mathrm{e}^z:=\sum_{k=0}^{\infty}\frac{z^k}{k!}\).
	这样我们就将\(f(x)=\mathrm{e}^x\)延拓到了\(\mathbb{C}\)上.
\end{definition}
\begin{lemma}\label{lem:2}
	全体代数数构成一个数域,即代数数的和、差、积、商仍为代数数.
\end{lemma}
至此基本的概念已经建立,下面不加证明地叙述这个重要的定理:
\begin{lemma}[林德曼-魏尔斯特拉斯定理]\label{lem:1}
	若 \(\alpha_1,\alpha_2,\cdots,\alpha_n \in \mathbb{C}\)都是代数数,且他们在 \(\mathbb{Q}\)上线性无关,
	则 \(\mathrm{e}^{\alpha_1},\mathrm{e}^{\alpha_2},\cdots,\mathrm{e}^{\alpha_n}\)在 \(\mathbb{Q}\)上代数无关.
\end{lemma}
准备工作完毕,下面我们来证明\(\pi\)是超越数.
\begin{proof}
	由指数函数的定义知\(\mathrm{e}^{\mathrm{i}\theta}:=\sum_{k=0}^{\infty}\frac{(\mathrm{i}\theta)^k}{k!}\).
	整理可得\(\mathrm{e}^{\mathrm{i}\theta}=\sum_{k=0}^{\infty}(-1)^k\frac{\theta^{2k}}{(2k)!}+\mathrm{i}\sum_{k=0}^{\infty}(-1)^k\frac{\theta^{2k +1}}{(2k +1)!}=\cos \theta + \mathrm{i} \sin \theta\).
	从而\(\mathrm{e}^{\mathrm{i}\pi}=\cos \pi + \mathrm{i}\sin \pi=-1\).
	故\(\mathrm{e}^{\mathrm{i}\pi}+1=0\),从而\(\mathrm{e}^{\mathrm{i}\pi}\)在\(\mathbb{Q}\)上是代数相关的.
	由引理\ref{lem:1}可知\(\mathrm{i}\pi\)不是代数数.
	又由引理\ref{lem:2}可知,若\(\pi\)是代数数,则由于\(\mathrm{i}^2+1=0\)知\(\mathrm{i}\)是代数数,从而\(\mathrm{i}\pi\)是代数数,矛盾!
	故\(\pi\)不是代数数,从而是超越数.
\end{proof}

\end{document}
