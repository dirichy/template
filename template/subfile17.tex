%!TEX TS-program = xelatex
%!language = zh
\documentclass[main]{subfiles}
%这是一个子文件,单独编译时会自动导入main文件的导言区
%这里可以放自定义命令,不会和别人的冲突请放心
%但是不能放newtheorem等高级命令,需要请在群里说
\begin{document}
\renewcommand{\filename}{No.17Theorem}%在这里填你的文件名,避免\label冲突
\section{超越数}
\subsection{背景介绍}
超越一词最早在莱布尼兹(Leibniz)1962年的一篇论文中出现,用来描述函数.
他证明了 \(\sin x\)是 \(x\)的超越函数.

现代超越数的概念最早由欧拉在18世纪提出.

1844年,刘维尔(Liouville)证明了超越数存在,并在1851年给出了第一个例子:
\(L_b:=\sum_{n=1}^{\infty} 10^{-n!}\).

1873年,埃尔米特(Hermite)证明了自然对数的底 \(\mathrm{e}\)是超越数.

1874年,康托(Cantor)证明了实数几乎是超越数,并给出了一个构造超越数的系统方法.

1882年,林德曼(Lindeman)证明了 \(\pi\)是超越数.

后来,又有许多数,如 \(\mathrm{e}^\pi,2^{\sqrt{2}},\sin 1,\ln a,e^b\)等,其中 \(a\)是不为 \(1\)的正有理数, \(b\)是不为 \(0\)的代数数.
\subsection{定理概述}
\begin{definition}\label{def:1}
  若一个复数 \(x \in \mathbb{C}\)不是任何一个非零整系数多项式的根,则称其为超越数.
  否则称其为代数数.
\end{definition}
\begin{theorem}\label{the:1}
  \(\sqrt{2}\)是代数数,所有有理数都是代数数。
\end{theorem}
\begin{proof}
  对于有理数\(\frac{p}{q}\)而言,我们令\(f(x)=qx-p\),则易知\(f(\frac{p}{q})=0\)。
  故\(\frac{p}{q}\)是整系数多项式\(f(x)=qx-p\)的根,从而是代数数。

  对于\(\sqrt{2}\)而言,我们令\(g(x)=x^2-2\),则有\(g(\sqrt{2})=0\)。
  故\(\sqrt{2}\)是有理系数多项式\(g(x)=x^2-2\)的根,从而也是代数数。
\end{proof}

代数数域是很大的数域,我们日常见到的很多数,如全体有理数,以及有有理数进行加减乘除开\(n\)次根号得到的数字都是代数数。
而且,以代数数为系数的多项式的根也都是代数数。
似乎代数数已经足够满足我们的需要了,我们目前的所有运算都在代数数域是封闭的。
那么,我们为什么需要超越数呢?
\begin{theorem}\label{the:2}
  自然对数的底\(\mathrm{e}\)是超越数。
\end{theorem}
数学家们先后发现\(\mathrm{e}\)不仅是无理数,还是超越数。
随后数学家们发现,代数数域对取极限是不封闭的,通过取极限的手段可以得到很多的超越数。

后来\(\pi\)被林德曼用代数学的方法证明是超越数。之后数学家们又陆续发现了许多超越数。
随着一个个超越数的发现,我们发现超越数是广泛存在的。
康托的结论更是说明了实数几乎全部都是超越数。

\subsection{定理意义}
超越数相关的证明,给数学带来了大的变革,解决了几千年来数学上的难题——尺规作图三大问题,即倍立方问题、三等分任意角问题和化圆为方问题。
随着超越数的发现,这三大问题被证明为不可能。
\end{document}
