%!TEX TS-program = xelatex
\documentclass[main]{subfiles}
%这是一个子文件,单独编译时会自动导入main文件的导言区
%这里可以放自定义命令,不会和别人的冲突请放心
%但是不能放newtheorem等高级命令,需要请在群里说
%下面是一些数学命令的简化,可以保留,可以删去,也可以按你的习惯修改
\def\e{\textup{e}}
\def\i{\textup{i}}
\def\T{\textup{T}}
\def\diag{\textup{diag}}
\def\id{\textup{id}}
\newcommand{\toi}[1]{{#1}\to\infty}
\newcommand{\dis}{\displaystyle}
\newcommand{\bv}{\mathrm{BV}}
\newcommand{\ac}{\mathrm{AC}}
\newcommand{\mr}{\mathbb{R}}
\newcommand{\mn}{\mathbb{N}}
\newcommand{\mq}{\mathbb{Q}}
\newcommand{\mz}{\mathbb{Z}}
\newcommand{\rel}{\text{ rel }}
\newcommand{\sgn}{\operatorname{sign}}
\newcommand{\ve}{\varepsilon}
\newcommand{\bs}{\backslash}
\newcommand{\Span}{\operatorname{span}}
\newcommand{\tr}{\mathrm{tr}}
\newcommand{\adj}{\mathrm{adj}}
\renewcommand{\ll}{\lim\limits}
\renewcommand{\ker}{\operatorname{Ker}}
\renewcommand{\hom}{\operatorname{Hom}}
\renewcommand{\leq}{\leqslant}
\renewcommand{\geq}{\geqslant}
\begin{document}
\renewcommand{\filename}{No.16Theorem}%在这里填你的文件名,避免\label冲突
%这里开始写你的代码
\section{Cayley–Hamilton 定理}
\subsection{定理内容}
在线性代数中,Cayley–Hamilton 定理表明对于任意交换环上的方阵(例如实方阵或复方阵)而言, 其特征多项式就是该矩阵的零化多项式.

对于交换环$R$上的$n\times n$的矩阵$\boldsymbol{A}$, 设$\boldsymbol{I}_n$是$R$上$n\times n$的单位矩阵, 那么$\boldsymbol{A}$的特征多项式是$p_{\boldsymbol{A}}(\lambda)=\det(\lambda \boldsymbol{I}_n-\boldsymbol{A})$, 其中$\det$表示取行列式, $\lambda\in R$是该多项式的变量. Cayley–Hamilton定理断言: $p_{\boldsymbol{A}}(\boldsymbol{A})=\boldsymbol{O}$, 其中$\boldsymbol{O}$表示零矩阵.

举一简单的例子以助理解: 假设$\boldsymbol{A}=\begin{pmatrix}
		1 & 2 \\
		3 & 4
	\end{pmatrix}$, 那么$\boldsymbol{A}$的特征多项式为
\[\begin{aligned}
		p_{\boldsymbol{A}}(\lambda)=\det(\lambda \boldsymbol{I}_2-\boldsymbol{A})
		 & =\begin{vmatrix}
			    \lambda-1 & -2        \\
			    -3        & \lambda-4
		    \end{vmatrix}                                  \\
		 & =(\lambda-1)(\lambda-4)-(-2)(-3)=\lambda^{2}-5\lambda-2
	\end{aligned}\]
Cayley–Hamilton定理声称: 将上式中的$\lambda$换成$\boldsymbol{A}$, 常数项用单位矩阵替换, 则一定有
\[p_{\boldsymbol{A}}(\boldsymbol{A})=\boldsymbol{A}^2-5\boldsymbol{A}-2\boldsymbol{I}_2=\boldsymbol{O}\]
真有这么巧吗? 让我们验算一下:
\[\boldsymbol{A}^2-5\boldsymbol{A}-2\boldsymbol{I}_2=\begin{pmatrix}
		7  & 10 \\
		15 & 22
	\end{pmatrix}
	-\begin{pmatrix}
		5  & 10 \\
		15 & 20
	\end{pmatrix}
	-\begin{pmatrix}
		2 & 0 \\
		0 & 2
	\end{pmatrix}
	=\begin{pmatrix}
		0 & 0 \\
		0 & 0
	\end{pmatrix}\]

\subsection{定理意义}
Cayley–Hamilton定理极大地\textbf{降低了寻找矩阵的零化多项式的难度}. 如果多项式$f(x)=a_{k}x^{k}+a_{k-1}x^{k-1}+\cdots+a_{1}x+a_0$使得$f(\boldsymbol{A})=a_{k}\boldsymbol{A}^{k}+a_{k-1}\boldsymbol{A}^{k-1}+\cdots+a_{1}\boldsymbol{A}+a_0\boldsymbol{I}_n=\boldsymbol{O}$, 就称多项式$f(x)$是矩阵$\boldsymbol{A}$的零化多项式. 零化多项式对于研究矩阵的代数性质有很大的帮助. 从前将$n$阶方阵全体看作一个$n^{2}$维的向量空间时, 我们只知道每个方阵都存在零化多项式, 并且次数最多是$n^2$; 有了Cayley–Hamilton定理之后我们发现特征多项式就是一个零化多项式, 次数也从原来的$n^2$降成了$n$. 对于阶数较高的方阵, 直接计算其特征多项式比较困难, 可结合牛顿恒等式($n=2$时就是初中时学过的韦达定理)计算其各项系数.

既然零化多项式已经容易找到, 我们可以\textbf{减少一些矩阵运算的计算量}. 仍以$\boldsymbol{A}=\begin{pmatrix}
		1 & 2 \\
		3 & 4
	\end{pmatrix}$为例,
经过计算其特征多项式我们得到
$\boldsymbol{A}^2-5\boldsymbol{A}-2\boldsymbol{I}_2=\boldsymbol{O}$,
那么$\boldsymbol{A}^2=5\boldsymbol{A}+2\boldsymbol{I}_2$,
也就是说计算$\boldsymbol{A}^2$转化为计算$\boldsymbol{A}$的线性组合了.
对于一般的$n$阶方阵$\boldsymbol{A}$,
这意味着计算$\boldsymbol{A}^{k}$可以化归为计算$\boldsymbol{A}^{k-1}$,
起到了降幂的作用. 计算矩阵的逆也变得更简单了: 仍以$\boldsymbol{A}=\begin{pmatrix}
		1 & 2 \\
		3 & 4
	\end{pmatrix}$为例, 由$\boldsymbol{A}^2-5\boldsymbol{A}-2\boldsymbol{I}_2=\boldsymbol{O}$得到$\dis\boldsymbol{A}\cdot\frac{1}{2}(\boldsymbol{A}-5\boldsymbol{I}_2)=\boldsymbol{I}_2$, 于是直接得到$\dis\boldsymbol{A}^{-1}=\frac{1}{2}(\boldsymbol{A}-5\boldsymbol{I}_2)$, 也就是说简化了逆矩阵的计算. 对于一般的$n$阶方阵也可以沿用相同的思路去计算矩阵的逆. 结合Cayley–Hamilton定理与多项式余式、Sylvester公式等知识还可以进行更多矩阵相关的巧妙计算.

Cayley–Hamilton定理的证明有诸多版本, 最简单的版本利用了伴随矩阵: 设$\boldsymbol{A}$的特征多项式是$p(t)$, 又设$\boldsymbol{B}:=\adj(t\boldsymbol{I}_n-\boldsymbol{A})$, 则根据伴随矩阵的定义有$(t\boldsymbol{I}_n-\boldsymbol{A})\boldsymbol{B}=\det(t\boldsymbol{I}_n-\boldsymbol{A})\boldsymbol{I}_n=p(t)\boldsymbol{I}_n$, 按$t$的幂次展开并比较该等式左右两侧系数, 再经过简单的计算即可得证. 从抽象代数的视角来看, 这实际上是建立了从矩阵环到矩阵多项式环的一个自然的环同构. 从Zariski拓扑的观点去看, Cayley–Hamilton定理还说明可对角化矩阵在全体方阵中是稠密的.
\end{document}
