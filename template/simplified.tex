%!TEX Ts-program = xelatex
% Options for packages loaded elsewhere
\PassOptionsToPackage{unicode}{hyperref}
\PassOptionsToPackage{hyphens}{url}
%
\documentclass[
]{ctexart}
\usepackage{amsmath,amssymb}
\usepackage{iftex}
\ifPDFTeX
  \usepackage[T1]{fontenc}
  \usepackage[utf8]{inputenc}
  \usepackage{textcomp} % provide euro and other symbols
\else % if luatex or xetex
  \usepackage{unicode-math} % this also loads fontspec
  \defaultfontfeatures{Scale=MatchLowercase}
  \defaultfontfeatures[\rmfamily]{Ligatures=TeX,Scale=1}
\fi
\usepackage{lmodern}
\ifPDFTeX\else
  % xetex/luatex font selection
\fi
% Use upquote if available, for straight quotes in verbatim environments
\IfFileExists{upquote.sty}{\usepackage{upquote}}{}
\IfFileExists{microtype.sty}{% use microtype if available
  \usepackage[]{microtype}
  \UseMicrotypeSet[protrusion]{basicmath} % disable protrusion for tt fonts
}{}
\makeatletter
\@ifundefined{KOMAClassName}{% if non-KOMA class
  \IfFileExists{parskip.sty}{%
    \usepackage{parskip}
  }{% else
    \setlength{\parindent}{0pt}
    \setlength{\parskip}{6pt plus 2pt minus 1pt}}
}{% if KOMA class
  \KOMAoptions{parskip=half}}
\makeatother
\usepackage{xcolor}
\setlength{\emergencystretch}{3em} % prevent overfull lines
\providecommand{\tightlist}{%
  \setlength{\itemsep}{0pt}\setlength{\parskip}{0pt}}
\setcounter{secnumdepth}{-\maxdimen} % remove section numbering
\ifLuaTeX
  \usepackage{selnolig}  % disable illegal ligatures
\fi
\IfFileExists{bookmark.sty}{\usepackage{bookmark}}{\usepackage{hyperref}}
\IfFileExists{xurl.sty}{\usepackage{xurl}}{} % add URL line breaks if available
\urlstyle{same}
\hypersetup{
  hidelinks,
  pdfcreator={LaTeX via pandoc}}

\author{}
\date{}
\renewcommand{\url}[1]{#1}

\begin{document}

\textbf{四平方和定理} ()
说明每个正整数均可表示为4个整数的平方和。它是费马多边形数定理和华林问题的特例。

\subsection{历史}\label{ux5386ux53f2}

\begin{itemize}
	\tightlist
	\item
		1743年,瑞士数学家欧拉发现了一个著名的恒等式:
\end{itemize}

\((a^2+b^2+c^2+d^2)(x^2+y^2+z^2+w^2)=(ax+by+cz+dw)^2+(ay-bx+cw-dz)^2+(az-bw-cx+dy)^2+(aw+bz-cy-dx)^2\)

根据上述欧拉恒等式或四元数的概念可知如果正整数\(m\)和\(n\)能表示为4个整数的平方和,则其乘积\(mn\)也能表示为4个整数的平方和。于是为证明原命题只需证明每个素数可以表示成4个整数的平方和即可。

\begin{itemize}
	\tightlist
	\item
		1751年,欧拉又得到了另一个一般的结果。即对任意奇素数 \(p\),同余方程
\end{itemize}

\(x^2+y^2+1 \equiv 0\pmod p\)
必有一组整数解x,y满足\(0 \le x<\frac{p}{2}\),\(0 \le y<\frac{p}{2}\)(引理一)

至此,证明四平方和定理所需的全部引理已经全部证明完毕。此后,拉格朗日和欧拉分别在1770年和1773年作出最后的证明。

\subsection{证明}\label{ux8b49ux660e}

根据上面的四平方和恒等式及算术基本定理,可知只需证明质数可以表示成四个整数的平方和即可。

\(2=1^2 + 1^2\),因此只需证明奇质数可以表示成四个整数的平方和。

根据引理一,奇质数\(p\)必有正倍数可以表示成四个整数的平方和。在这些倍数中,必存在一个最小的。设该数为\(m_0 p\)。又从引理一可知\(m_0 < p\)。

\subsubsection{\texorpdfstring{证明\(m_0\)不会是偶数}{证明m\_0不会是偶数}}\label{ux8b49ux660em_0ux4e0dux6703ux662fux5076ux6578}

设\(m_0\)是偶数,且\(m_0 p = x_1^2 + x_2^2 + x_3^2 + x_4^2\)。由奇偶性可得知必有两个数或四个数的奇偶性相同。不失一般性设\(x_1,x_2\)的奇偶性相同,\(x_3,x_4\)的奇偶性相同,\(x_1+x_2,x_1-x_2,x_3+x_4,x_3-x_4\)均为偶数,可得出公式:

\(\frac{m_0 p}{2} = \left(\frac{x_1+x_2}{2}\right)^2 + \left(\frac{x_1-x_2}{2}\right)^2 + \left(\frac{x_3+x_4}{2}\right)^2 + \left(\frac{x_3-x_4}{2}\right)^2\)

\(\frac{m_0}{2} < m_0\),与\(m_0\)是最小的正整数使得的假设\(m_0 p\)可以表示成四个整数的平方和不符。

\subsubsection{\texorpdfstring{证明
		\(m_0 = 1\)}{证明 m\_0 = 1}}\label{ux8b49ux660e_m_0_1}

现在用反证法证明\(m_0 = 1\)。设\(m_0 > 1\)。

\begin{itemize}
	\tightlist
	\item
		\(m_0\)不可整除\(x_i\)的最大公因数,否则\(m_0^2\)可整除\(m_0 p\),则得\(m_0\)是\(p\)的因数,但\(1 < m_0 < p\)且p为质数,矛盾。
\end{itemize}

故存在不全为零、绝对值小于\(\frac{1}{2} m_0\)(注意\(m_0\)是奇数在此的重要性)整数的\(y_1,y_2,y_3,y_4\)使得
\(y_i = x_i \pmod{m_0}\)。

\begin{description}
	\tightlist
	\item[]
		\(0 < \sum y_i^2 < 4 (\frac{1}{2} m_0 )^2 = m_0^2\)

		\(\sum y_i^2 \equiv \sum x_i^2 \equiv 0 \pmod{m_0}\)
\end{description}

可得 \(\sum y_i^2  = m_0 m_1\),其中\(m_1\)是正整数且小于\(m_0\)。

\begin{itemize}
	\tightlist
	\item
		下面证明\(m_1 p\)可以表示成四个整数的平方和,从而推翻假设。
\end{itemize}

令\(\sum z_i^2 = \sum y_i^2 \times \sum x_i^2\),根据四平方和恒等式可知\(z_i\)是\(m_0\)的倍数,令\(z_i = m_0 t_i\),

\begin{description}
	\tightlist
	\item[]
		\(\sum z_i^2 = \sum y_i^2 \times \sum x_i^2\)
		\(m_0^2 \sum t_i^2 = m_0 m_1 m_0 p\)
		\(\sum t_i^2 = m_1 p < m_0 p\)
\end{description}

矛盾。

\subsubsection{引理一的证明}\label{ux5f15ux7406ux4e00ux7684ux8b49ux660e}

将和为\(p-1\)的剩余两个一组的分开,可得出\(\frac{p+1}{2}\)组,分别为\((0,p-1), (1,p-2) , ... , ( \frac{p-1}{2}, \frac{p-1}{2})\)。
将模\(p\)的二次剩余有\(\frac{p+1}{2}\)个,分别为\(0,1^2,2^2,...,(\frac{p-1}{2})^2\)。

若\(\frac{p-1}{2}\)是模\(p\)的二次剩余,选取\(x< \frac{p}{2}\)使得\(x^2 \equiv \frac{p-1}{2}\),则\(1 + x^2 + x^2 \equiv 0 \pmod{p}\),定理得证。

若\(\frac{p-1}{2}\)不属于模\(p\)的二次剩余,则剩下\(\frac{p-1}{2}\)组,分别为\((0,p-1), (1,p-2) , ... , ( \frac{p-3}{2}, \frac{p+1}{2} )\),而模\(p\)的二次剩余仍有\(\frac{p+1}{2}\)个,由于
\(\frac{p+1}{2} > \frac{p-1}{2}\)
,根据抽屉原理,存在\(1 + x^2 + y^2 \equiv 0 \pmod{p}\)。

\end{document}
